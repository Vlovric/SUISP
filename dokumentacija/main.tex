\documentclass[]{foi}
\usepackage{lipsum}
\usepackage[utf8]{inputenc}
\usepackage{float}

\vrstaRada{\seminar}
\predmet{Strate\v{s}ko upravljanje informacijskom sigurno\v{s}\'{c}u i privatno\v{s}\'{c}u}
\title{Klijentski kriptirani trezor datoteka s revizijskim tragom u Pythonu}

\author{Viktor Lovrić, Magdalena Markovinović, David Matijanić}
\spolStudenta{\musko}
\mentor{Izv. prof. dr. sc. Igor Tomičić}
\spolMentora{\musko}
\godina{2025}
\mjesec{Listopad}
\date{2025}
\status{redoviti}
\indeks{0016154953, 0016155896, 0016153844}
\smjer{Informacijsko i programsko inženjerstvo}
\titulaProfesora{}

\sazetak{Ovaj rad prolazi kroz teoriju i implementaciju sigurnosnog kriptiranog trezora s revizijskim tragom koristeći Python. Sigurnosni višeplatformski kriptirani trezor datoteka s enkripcijom podataka u mirovanju i potpunom mogućnošću revizije mora biti usklađen s kriptografskim i \textit{logging} kontrolama prema ISO 27001.}

\kljucneRijeci{datoteke, enkripcija, trezor, revizijski trag, Python, ključ}

\begin{document}

\maketitle

\tableofcontents

\pagestyle{plain}

\chapter{Uvod}

Digitalni trezori ili sigurna spremišta podataka važan su alat za zaštitu osjetljivih informacija, kako privatnih, tako i onih poslovnih. Moderni sustavi pohrane koriste razne načine upravljanja pristupa i zaštite podataka, ali ne osiguravaju uvijek adekvatnu razinu sigurnosti i privatnosti. 
Lako su ranjivi na različite prijetnje, uključujući neovlašteni pristup, krađu podataka i gubitak podataka zbog tehničkih kvarova ili ljudskih pogrešaka. Kriptografska enkripcija podataka pruža dodatnu razinu zaštite. Ovaj rad želi prikazati složenost implementacije jednog takvog sustava, fokusirajući se na ključne aspekte kao što su slojevita enkripcija, kontrola pristupa, revizija, životni ciklus ključa, sigurnosno brisanje, sigurnosno dijeljenje i mehanizmi zaključavanja ili odgode.

\section{ISO 27001}

ISO 27001 međunarodni je standard koji definira zahtjeve za sustav upravljanja informacijskom sigurnošću (engl. \textit{Information Security Management System, ISMS}). Ovaj standard pruža okvir za zaštitu povjerljivosti, cjelovitosti i dostupnosti informacija unutar organizacije. Implementacija ISO 27001 standarda pomaže organizacijama u identifikaciji i upravljanju rizicima vezanim uz informacije te osigurava usklađenost s pravnim i regulatornim zahtjevima.
Ključan dio ISO 27001 standarda za ovaj projekt su \textbf{kriptografske kontrole} definirane u sekciji \textbf{A.10} (Politika korištenja kriptografskih kontrola A.10.1.1 i Upravljanje ključevima A.10.1.2). One uključuju zahtjeve za upravljanje kriptografskim ključevima, zaštitu podataka tijekom prijenosa i pohrane te osiguranje integriteta i povjerljivosti informacija. 

\textbf{Kontrole pristupa i autentikacije (A.9.4.2 i A.9.4.3)} zahtjevaju definiranje i upravljanje pravima pristupa korisnika te implementaciju sigurnih metoda autentikacije prije pristupanja resursima. Ove kontrole direktno se odnose na implementaciju lozinki i autentikacijskih mehanizama temeljenih na ključevima. 

\textbf{Kontrola A.11.2.7} (Sigurno odlaganje ili ponovna upotreba opreme) zahtijeva da sva oprema koja sadrži medije za pohranu bude provjerena kako bi se osiguralo da su svi podaci uklonjeni ili zaštićeni, što projekt adresira kroz \textit{secure deletion workflow} koji osigurava kriptografski siguran način brisanja podataka i ključeva.

\textbf{Kontrole zapisivanja i zaštite zapisa revizije (A.12.4.1, A.12.4.2 i A.12.4.3)} zahtijevaju izradu zapisnika događaja koji bilježe aktivnosti korisnika, iznimke i greške u sustavu, te zaštitu tih zapisa od neovlaštenih promjena. Projekt implementira nepromjenjivi zapisnik revizije koji evidentira sve operacije s ključevima, rotacije, pristupe datotekama te administratorske aktivnosti. Zapisnici su zaštićeni od izmjena i omogućuju export za eksterne revizije, čime se osigurava transparentnost i mogućnost revizije sigurnosnih događaja u skladu s regulatornim zahtjevima.
\cite{iso_27001}

\chapter{Mehanizmi potrebni za implementaciju}

U ovom su dijelu rada opisani svi mehanizmi potrebni za implementaciju klijentskog kriptiranog trezora s revizijskim tragom koji je u skladu s ISO 27001 standardom.
U kasnijem dijelu rada ovi su mehanizmi pretvoreni u funkcionalne i nefunkcionalne zahtjeve te i implementirani.


\section{Slojevita enkripcija (engl. \textit{envelope encryption})}

Slojevita enkripcija je proces kriptiranja ključa drugim ključem \cite{google_envelope}.
\newline
Ključ za kriptiranje podataka (engl. \textit{data encryption key}), u nastavku teksta \textbf{DEK}, je simetrični ključ koji se koristi za kriptiranje i dekriptiranje podataka jedne datoteke jednog korisnika.
Za svaku datoteku se koristi zaseban DEK koji se nakon generiranja kriptira pri pohrani što zadovoljava zahtjev kriptiranja pri pohrani (engl. \textit{encryption at rest}) \cite{google_envelope}.
\newline
Ključ za kriptiranje ključa (engl. \textit{key encryption key}), u nastavku teksta \textbf{KEK}, je asimetrični javni ključ koji koji se koristi za \textit{omotavanje} (engl. \textit{wrapping}) DEK-a, tj. kriptiranje DEK-a \cite{google_envelope}.
KEK se generira prilikom kreiranja korisničkog računa koristeći RSA (engl. \textit{Rivest-Shamir-Adleman}) ili ECC (engl. \textit{Elliptic Curve Cryptography}) algoritme i koristi se kako bi jedan KEK kriptirao više DEK-ova što eliminira potrebu za rotiranjem ključeva za svaku datoteku, već je potrebno rotirati samo KEK.
Za generiranje DEK-a je preporučeno korištenje AES-256 (engl. \textit{Advanced Encryption Standard}) algoritma u GCM (engl. \textit{Galois Counter Mode}) načinu rada zbog njegove sigurnosti i brzine \cite{google_envelope}.
AES-256-GCM pruža i efikasnu provjeru integriteta podataka koristeći autentifikacijsku oznaku (engl. \textit{authentication tag}) kojom prilikom dekriptiranja provjerava integritet podataka.


\section{Kontrola pristupa i revizija}

Kontrola pristupa se sastoji od autentifikacije i autorizacije korisnika. Autentifikacija je proces u kojemu sustav provjerava identitet korisnika kako bi mu omogućio pristup \cite{ibm_access}, što se radi kroz jedno od 3 metode: nešto što korisnik zna (npr. lozinka), nešto što korisnik ima (npr. token) ili nešto što korisnik jest (npr. biometrijski podaci).
Autorizacija je proces određivanja razine pristupa resursima sustava na temelju korisničkog identiteta i njegovih dopuštenja \cite{ibm_access}. S obzirom da je naš sustav klijentski i zamišljen samo za jednog korisnika po uređaju, autorizacija se neće obavljati već će jedinstveni korisnik imati potpune ovlasti.
Odabrana metoda autentifikacije je lozinka iz koje se generira ključ koristeći neki od algoritama za derivaciju ključa poput PBKDF2 (engl. \textit{Password-Based Key Derivation Function 2}), scrypt ili Argon2 kako bi se taj ključ mogao pohraniti za autentifikaciju, a da pritom ostane zaštićen \cite{kdf}.
\newline
Revizija je proces pregleda i preispitivanja procesa i stanja kako bi se prosudila razina usklađenosti stanja s unaprijed definiranim kriterijima \cite{revizija}.
Kako bi se revizija mogla provesti, potrebno je voditi revizijski trag (engl. \textit{audit log}) svih važnih događaja i radnji u sustavu. Važna osobina revizijskog traga je njegova nepromjenjivost kako bi se spriječile zloupotrebe i manipulacije podacima \cite{audit}. Iz tog razloga se za validaciju ispravnosti revizijskog traga koriste mehanizmi poput digitalnog potpisa i lanca sažetaka (engl. \textit{hash chain}) gdje se svaki novi zapis povezuje sa sažetkom prethodnog zapisa, čime se osigurava da bilo kakva izmjena u prethodnim zapisima rezultira promjenom u sažetcima i time postaje vidljiva, što je ista tehnika koja se koristi u blockchain sustavima.
Još jedna važna karakteristika revizijskog traga je što on mora biti izvoziv (engl. \textit{exportable}) kako bi se mogao analizirati izvan samog sustava, npr. u slučaju inspekcije od strane treće strane.
U tom je slučaju potrebno revizijski trag kriptirati javnim ključem osobe koja će vršiti inspekciju kako bi se osigurala povjerljivost podataka.

\section{Životni ciklus ključa}

Ovaj odlomak opisuje neke važne karakteristike svake faze životnog ciklusa ključa. Svaka od faza se bilježi u revizijski trag, kao i neke dodatne radnje.
\newline
Životni ciklus ključa se sastoji od nekoliko faza koje su važne za ovaj sustav \cite{key_lifecycle}:
\begin{itemize}
    \item \textbf{Generiranje ključa} - Obavezno korištenje sigurnog algoritma i generatora nasumičnih brojeva za generiranje ključa.
    \item \textbf{Pohrana ključa} - Spremanje ključa se radi samo u kriptiranom obliku.
    \item \textbf{Rotacija ključa} - Redovita rotacija ključa smanjuje rizik od komprimiranih ključeva. Rotacija ključa, kao i standardi za generiranje ključa mogu biti definirani u sigurnosnoj politici koju sustav zatim implementira i prati.
    \item \textbf{Arhiviranje i uništavanje ključa} - Kada ključ više nije potreban, potrebno ga je sigurno uništiti kako se ne bi mogao iščitati ili koristiti.
\end{itemize}

\section{Sigurnosno brisanje}

Iz razloga što klasične metode brisanja ne uklanjaju podatke s medija već samo označavaju prostor kao slobodan za prepisivanje, potrebno je implementirati sigurnosno brisanje (engl. \textit{secure deletion}) kako bi se spriječilo vraćanje izbrisanih podataka. Sigurnosno brisanje se može implementirati na nekoliko načina \cite{secure_deletion}, no za ovaj sustav najbolje implementacije su kriptografsko brisanje (engl. \textit{cryptographic erasure}) i prepisivanje (engl. \textit{overwriting}). Kriptografsko brisanje se sastoji od uništavanja ključa koji je korišten za kriptiranje podataka, čime kriptirani podatci postaju nečitljivi. Prepisivanje se sastoji od višestrukog prepisivanja podataka s nasumičnim podatcima kako bi se spriječilo njihovo iščitavanje.

\section{Sigurnosno dijeljenje}

Sigurnosno dijeljenje podrazumijeva E2EE (engl. \textit{end-to-end encryption}) dijeljenje podataka između korisnika bez izlaganja podataka trećim stranama. To se može postići korištenjem Shamirovog tajnog dijeljenja (engl. \textit{Shamir's Secret Sharing}) ili dijeljenih ključeva specifičnih za svaku datoteku, pri čemu se ključ sadržaja ponovno kriptira javnim ključem svakog primatelja (engl. \textit{per-recipient re-wrapping}). Za ovaj sustav odabrana je druga metoda zbog klijentske prirode sustava i jednostavnosti implementacije. Ovom implementacijom je omogućeno da datoteka može biti dekriptirana sa više privatnih ključeva jer je njen DEK kriptiran s više javnih ključeva.

\section{Mehanizam zaključavanja ili odgode}

Zadnji mehanizam koji je potreban za implementaciju je mehanizam zaključavanja (engl. \textit{lockout}) i mehanizam odgode (engl. \textit{backoff}) gdje oba mehanizma služe za obranu od brute-force napada na lozinku.
\newline
Mehanizam zaključavanja onemogućava pristup sustavu nakon određenog broja neuspješnih pokušaja prijave, dok mehanizam odgode povećava vrijeme čekanja između pokušaja prijave nakon svakog neuspješnog pokušaja. Nužno je balansirati sigurnost i korisničko iskustvo jer svaki mehanizam ima svoje mane. Mehanizam zaključava se može činiti frustrirajućim za korisnika, dok mehanizam odgode može biti negiran ako napadač koristi višedretveni napad neke vrste \cite{brute-force}.

\chapter{Funkcionalnosti}

Ovo poglavlje opisuje sve funkcionalne i nefunkcionalne zahtjeve za implementaciju klijentskog kriptiranog trezora s revizijskim tragom koji prati ISO 27001 standard.
Funkcionalni zahtjevi su sljedeći:
\begin{enumerate}
    \item \textbf{Postavljanje inicijalne lozinke sustava} - Na početku korištenja softvera, korisnika se traži da postavi lozinku (engl. \textit{passphrase}) koju će koristiti. Lozinka će se morati mijenjati nakon nekog vremena.
    \item \textbf{Prijava i odjava iz sustava} - Koristeći trenutnu lozinku, korisnik se može prijaviti u sustav i moguće je odjaviti se.
    \item \textbf{Prijenos datoteke u kriptirani trezor} - Jedna od ključnih funkcionalnosti, korisnik mora moći na jednostavan način prenijeti datoteku u kriptirani trezor, koja će zatim biti sigurno pohranjena.
    \item \textbf{Dohvaćanje datoteke iz kriptiranog trezora} - Korisnik mora moći koristeći svoju lozinku dohvatiti datoteku iz kriptiranog trezora.
    \item \textbf{Sigurnosno brisanje datoteke iz kriptiranog trezora} - Kako "obrisane" datoteke i dalje postoje na disku i može ih se pročitati, sustav mora omogućiti sigurnosno brisanje datoteke, što bi uklonilo i taj ostatak, ili bi samo uništilo ključ, kako bi datoteka ostala zauvijek kriptirana.
    \item \textbf{Nepromjenjivi revizijski trag s mogućnošću izvoza} - Sustav sve radnje, kao što su pokušaji prijave ili odjave, prijenos datoteka, čitanje i brisanje, treba bilježiti kao nepromjenjivi revizijski trag. Na taj način sve se radnje mogu pratiti kako bi se mogli detektirati pokušaji upada ili stvarni upadi.
    \item \textbf{Sigurnosno dijeljenje datoteka} - Korisnik treba moći dijeliti datoteku na siguran način.
\end{enumerate}

Kako za sigurnost softvera vrijedi: "ako se za ono što sustav nije namijenjen ne dogodi" \cite{sis_uvodna_preza}, nije dovoljno da softver ispunjava sve funkcionalne zahtjeve, nego je iznimno bitno da ispunjava i nefunkcionalne zahtjeve. Nefunkcionalni zahtjevi su sljedeći:
\begin{enumerate}
    \item \textbf{Softver treba biti u skladu s ISO 27001 standardima za kriptografske kontrole} - Svi gore standardi moraju biti pokriveni. Sljedeći nefunkcionalni zahtjevi samo proširuju i detaljnije objašnjavaju na što se to odnosi.
    \item \textbf{Softver treba upozoravati na slabe lozinke} - Softver bi trebao upozoravati na lozinke koje nisu u skladu s pravilima za lozinku i tražiti mijenjanje lozinki nakon određenog vremena.
    \item \textbf{Softver ne bi trebao dopustiti više od par pokušaja prijave s \textit{backoff} i \textit{lockout} mehanizmom} - Svaki novi unos lozinke nakon pogrešno unesene bi trebao trajati duže za provjeru je li ispravna. Tako se onemogućuju \textit{brute-force} pokušaji provale lozinke. Također, nakon određenog broja pokušaja, sustav se treba zaključati na određeno vrijeme, da se lozinka ni ne može unositi.
    \item \textbf{Sustav ne bi nikada trebao pohranjivati datoteku u \textit{plaintext} formatu} - Svi podaci u mirovanju (engl. \textit{data at rest}) moraju biti kriptirani. Podaci u mirovanju su "neaktivni podaci koji se trenutno ne premještaju između mreža ili uređaja" \cite{data-at-rest}, odnosno nikakva radnja se ne obavlja nad njima.
    \item \textbf{Sigurnosni detalji ne bi trebali biti javno vidljivi} - Softver treba biti siguran i koristiti sve sigurnosne mehanizme i mehanizme kriptiranja, bez da korisnik mora znati kako su implementirani. Drugim riječima, softver bi trebao biti jednostavan za korištenje, bez obzira što se u pozadini obavljaju najstrože sigurnosne funkcije.
\end{enumerate}

\section{Tehnologije za implementaciju}

Softver će biti implementiran koristeći programski jezik Python. Bit će klijentski (bez poslužitelja), odnosno izvršavat će se na jednom računalu za jednog korisnika. Imat će korisničko sučelje (GUI). Za kriptografske funkcionalnosti koristit će se već poznate Python biblioteke.

\chapter{Implementacija}

\section{Prijava, registracija i odjava}

Implementacija aplikacije započinje izradom registracije odnosno postavljanjem inicijalne lozinke u sustav. Aplikacija prepozaje ako ne postoji niti jedan korisnik u sustavu te upućuje korisnika na ekran registracije. Ako korisnik već postoji u bazi, aplikacija korisnika upućuje na ekran prijaveKVažno je napomenuti da se do registracije ne mođe doći niti jednim drugim načinom osim ako u bazi ne postoji korisnik, stoga se aplikacija može korisititi isklučivo od strane jednog korisnika koji je ujedno i vlasnik profila na računalu. 

\begin{figure}[h]

\begin{subfigure}
\includegraphics[width=0.5\linewidth, height=6cm]{slike/prijava_suisp.png} 
\end{subfigure}
\begin{subfigure}
\includegraphics[width=0.5\linewidth, height=6cm]{slike/registracija_suisp.png}
\end{subfigure}

\caption{Zaslon prijave i registracije}
\end{figure}

Kako bi lozinka bila sigurna, prilikom unosa se provjeravaju njezina jačina i složenost po slijedećim parametrima:
   \textbf{minimalna duljina od 12 znakova}, \textbf{barem jedno veliko slovo}, \textbf{barem jedno malo slovo}, \textbf{barem jedan broj} i \textbf{barem jedan specijalni znak}.
Aplikacija provjerava i minimalnu entropiju lozinke koja po ne smije niti manja od 80 bitova. Ukoliko lozinka ne zadovoljava navedene uvjete, korisnik dobiva povratnu informaciju o tome što treba popraviti. 

Lozinka se, nakon upisa, hashira koristeći \textbf{PBKDF2 algoritam} s \textbf{SHA-256} funkcijom sažetka, \textbf{600000 iteracija}. Algoritam kao parametre prima Master key (objašnjeno u nastavku) i lozinku u obličnom tekstu kao sol.
Hash lozinke se sprema se u bazu podataka.

Odjava iz sustava vrši se pritiskom na gumb odjave koji briše sve osjetljive podatke (KEK, PDK) iz memorije (sesije) i vraća korisnika na ekran prijave.
Sve akcije (registracija, prijava i odjava) se bilježe u revizijski trag u bazi podataka.

\subsection{Dodatni zaštitni mehanizmi prijave i dojave}

Kako bi se spriječili \textit{bruteforce} napadi na lozinku, implementirani je \textit{lockout} mehanizam. Nakon 3 nesupjela pokušaja prijave, vrijeme čekanja za sljedeća 3 pokušaja povećava za minutu (npr. 3 neuspjela = 60s, 6 neuspjelih = 120s...). Važno je napomenuti da nema gornje granice za broj pokušaja. Kako bi se spriječilo rsetiranje vremena čekanja ako se korisnik ugasi aplikaciju ili računalo, vrijeme posljednjeg neuspjelog pokušaja prijave se sprema u bazu podataka i koristi se za izračun vremena čekanja prilikom svakog novog pokušaja prijave.

\begin{figure}[h]

\begin{subfigure}
\includegraphics[width=0.5\linewidth, height=6cm]{slike/odbrojavanje_60.png} 
\end{subfigure}
\begin{subfigure}
\includegraphics[width=0.5\linewidth, height=6cm]{slike/odbrojavanje_120.png}
\end{subfigure}

\caption{Primjer lockout mehanizma nakon 3 i 6 neuspjelih pokušaja prijave}
\end{figure}

Drugi mehanizam zaštite je automatska odjava iz sustava nakon određenog vremena neaktivnosti (15 minuta). Nakon isteka tog vremena, prikazuje se upozorenje o odjavi iz sustava od dodatnih 60 sekundi. Ako korisnik ne reagira u zadano vrijeme, sustav ga automatski odjavljuje.

\begin{figure}[h]
    
\includegraphics[width=1\linewidth, height=6cm]{slike/automatska_odjava.png}
\caption{Zaslon automatske odjave nakon neaktivnosti}
\end{figure}

\subsection{Generiranje javnog i privatnog ključa}

Nakon uspješne registracije, sustav generira par ključeva (javni i privatni) koristeći \textbf{RSA algoritam} s duljinom ključa od \textbf{4096 bita}. 
Prije kriptiranja privatnog ključa generiraju se dva međuključa pomoću \textbf{PBKDF2 algoritma}. U prvoj iteraciji, generira se \textit{Master key} (dalje u teksu MK) sa velikim brojem iteracija (6000000) nakon kojega se generira \textit{Password Derived Key} (dalje u tekstu PDK) s manjim brojem iteracija (1) koji se koristi za kriptiranje privatnog ključa. PDK ključ se NE sprema u bazu nego ostaje samo u memoriji tijekom trajanja sesije.
Privatni ključ se zatim kriptira \textbf{AES-256-GCM algoritmom} koristeći PDK generiran iz lozinke.
Na kraju se  javni ključ, kriptirani privatni ključ i hash lozinke spremaju u bazu.

\subsection{Cleananje datoteka}

Viktor

\subsection{Jedna instanca}

Viktor

\section{Sigurnosna politika i rotacija ključeva}

Viktor

\section{Prijenos datoteke}

Implementirana je mogućnost prijenosa datoteke u siturnosni kriptirani trezor. To je moguće učiniti na zaslonu "Zaključane datoteke" klikom na gumb "Prenesi".
Sam proces je jednostavan i intuitivan za razumjeti, što odgovara funkcionalnom zahtjevu \textbf{5. Sigurnosni detalji ne bi trebali biti javno vidljivi}.
Nakon prijenosa datoteke, ona je vidljiva na popisu svih zaključanih datoteka. Sljedeća slika prikazuje kako izgleda zaslon zaključanih datoteka nakon što su se određene prenijele:

\begin{figure}[H]
    \centering
    \includegraphics[width=1\textwidth]{slike/prenesene_datoteke.png}
    \caption{Datoteke su prenesene u sustav.}
\end{figure}

Kako datoteka još nije prošla otključavanje i ponovno zaključavanje, ispod nazive piše vrijeme prijenosa, a ne vrijeme ažuriranja.

Implementacija u pozadini aplikacije obavlja sljedeće korake:

\begin{enumerate}
    \item Dobiva se datoteka koju je odabrao korisnik.
    \item Dohvaća se korisnikov javni KEK i generira se nasumični DEK.
    \item DEK se enkriptira javnim KEK-om koristeći RSA algoritam.
    \item Datoteka se kriptira DEK-om koristeći AES algoritam.
    \item Datoteka se pohranjuje u određeni direktorij i u bazu podataka se sprema zapis koji sadrži izvorni naziv datoteke, putanju do kriptirane datoteke, kriptirani DEK i ostale informacije.
\end{enumerate}

Iako Python sadrži sakupljač smeća (engl. \textit{garbage collector}), za svaki slučaj je na kraju izvršenja metode za prijenos datoteke dodatno obrisan sadržaj učitane datoteke iz memorije, tako što je prepisan s nulama i zato eksplicitno obrisan koristeći ključnu riječ \texttt{del}:
\begin{listing}[H]
    \begin{minted}[frame=lines, linenos, breaklines]{python}
file.content = b'\x00' * len(file.content)
del file
    \end{minted}
    \caption{Uklanjanje sadržaja učitane datoteke iz memorije.}
\end{listing}

Kriptirane datoteke se mogu vidjeti u direktoriju, no ne mogu se pročitati bez DEK-a:

\begin{figure}[H]
    \centering
    \includegraphics[width=0.4\textwidth]{slike/kriptirane_datoteke.png}
    \caption{Kriptirane datoteke u sustavu.}
\end{figure}

\section{Zaključavanje i otključavanje datoteke}

Viktor

\section{Brisanje datoteke}

Pritiskom gumba za brisanje na odabranoj datoteci, poziva se funkcija koja prvo traži postoji li odabrana kriptirana datoteka na našem računalu. Ako postoji, datoteka se prvo otvara, prepisuje tako što se upišu nulti bajtovi preko cijele datoteke, zatim se zatvara i briše iz datotečnog sustava. Nakon brisanja datoteke na računalu, briše se i zapis o datoteci iz baze podataka. Sve ove radnje bilježe se u revizijski trag. 

Implementirana je zaštita od pogrešnog prisanja datoteka tako što se pritiskom na gumb otvara dijalog prozor koji traži potvrdu brisanja datoteke. KAo potvrda za brisanje, korisnik mora unjeti naziv datoteke koju želi izbrisati. Ako uneseni naziv odgovara nazivu datoteke koja se briše, datoteka se briše, inače se ispisuje greška.

\begin{figure}[h]

\begin{subfigure}
\includegraphics[width=0.5\linewidth, height=6cm]{slike/dijalog_brisanje_datoteke.png} 
\end{subfigure}
\begin{subfigure}
\includegraphics[width=0.5\linewidth, height=6cm]{slike/dijalog_brisanje_greska.png}
\end{subfigure}

\caption{Zaslon za potvrdu brisanja datoteke i prikaz greške pri pogrešnom unosu}
\end{figure}

\pagebreak

\section{Dijeljenje datoteke}

Implementirana je funkcionalnost sigurnosnog dijeljenja datoteke. Na desnom kraju svake zaključane datoteke na popisu "Zaključane datoteke" nalazi se gumb za dijeljenje.
Klikom na taj gumb otvara se zaslon na kojemu je moguće unijeti javni ključ osobe kojoj želimo podijeliti datoteku. Sljedeća slika pokazuje taj zaslon:
\begin{figure}[H]
    \centering
    \includegraphics[width=1\textwidth]{slike/zaslon_za_dijeljenje_datoteke.png}
    \caption{Zaslon za dijeljenje datoteke.}
\end{figure}

Nakon unosa javnog ključa i klika na gumb "Podijeli", dijeljena datoteka se izvozi s ekstenzijom \textit{.shfipkg} (\textit{shared file package}):

\begin{figure}[H]
    \centering
    \includegraphics[width=0.2\textwidth]{slike/podijeljena_datoteka.png}
    \caption{Podijeljena datoteka.}
\end{figure}

Osoba koja želi učitati dijeljenu datoteku, \textbf{prvo mora osobi koja dijeli datoteku poslati svoj javni ključ}. To je moguće učiniti na zaslonu "Prijenos dijeljene datoteke":

\begin{figure}[H]
    \centering
    \includegraphics[width=1\textwidth]{slike/zaslon_za_prijenos_dijeljene_datoteke.png}
    \caption{Zaslon za prijenos dijeljene datoteke.}
\end{figure}

Na slici je vidljiv gumb "Kopiraj javni ključ" koji kopira javni ključ te osobe, odnosno javni dio KEK-a u međuspremnik. Na dnu zaslona se nalazi gumb "Učitaj datoteku" koji otvara odabir dijeljene datoteke.
Nakon odabira datoteke s ekstenzijom \textit{.shfipkg}, ako je podijeljena koristeći ispravni javni ključ, datoteka se unosi u kriptirani trezor.

Iako je dijeljenje ove datoteke intuitivno, aplikacija u pozadini obavlja dodatne radnje, kako bi se osigurala sigurnost datoteke u prijenosu te osigurali povjerljivost i integritet.

Na strani osobe koja dijeli datoteku obavlja se sljedeće:
\begin{enumerate}
    \item Dohvaća se zalijepljen javni ključ osobe (javni dio KEK-a) kojoj se želi podijeliti datoteka, kako bi se DEK mogao zaključati istim.
    \item Učitava se kriptirani sadržaj datoteke koja se želi podijeliti.
    \item Dohvaća se privatni ključ osobe (privatni dio KEK-a) koja dijeli datoteku.
    \item DEK pripadajuće datoteke se dekriptira privatnim ključem osobe koja dijeli datoteku i ponovno kriptira javnim ključem osobe kojoj se dijeli datoteka.
    \item Kreira se datoteka koja sadrži kriptiranu datoteku, kriptirani DEK i dodatne metapodatke (kao što su originalni naziv datoteke i informacija o tome je li binarna ili ne).
\end{enumerate}
Na ovaj način omogućeno je da nitko ne može pročitati sadržaj datoteke bilo gdje u prijenosu od pošiljatelja do primatelja, pošto nema privatni ključ osobe kojoj se datoteka šalje.
Isto tako, nije moguće pročitati dekriptiranu datoteku iz memorije jer se nikad ne dekriptira. Još jedna prednost ovog pristupa s KEK-om i DEK-om je ta što je proces dekripcije i enkripcije DEK-a iznimno brz te u slučaju da je kriptirana datoteka ogromna, nije ju potrebno cijelu dekriptirati i enkriptirati ponovno.

Na strani osobe koja učitava dijeljenu datoteku obavlja se sljedeće:
\begin{enumerate}
    \item Učitava se paket s dijeljenom datotekom u sustav.
    \item Pokušava se dekriptirati kriptirani DEK iz dijeljene datoteke, kako bi se utvrdilo da je kriptiran s ispravnim javnim ključem, odnosno javnim dijelom KEK-a osobe koja učitava datoteku te u slučaju da to nije moguće, ne nastavlja se s učitavanjem datoteke u sustav.
    \item Ako je DEK uspješno dekriptiran, pokušava se dekriptirati datoteku kako bi se utvrdilo da je podijeljena odgovarajuća datoteka za taj DEK.
    \item Ako je sve uspješno obavljeno, kriptirana datoteka se ubacuje u direktorij s ostalim kriptiranim datotekama i dodaje se novi zapis u bazi koji sadrži putanju, kriptiran DEK i ostale informacije.
\end{enumerate}

Na kraju se iz memorije brišu dekriptirani DEK i dekriptirana datoteka na sljedeći način:
\begin{listing}[H]
    \begin{minted}[frame=lines, linenos, breaklines]{python}
decrypted_dek = b'\x00' * len(decrypted_dek)
del decrypted_dek

decrypted_file = b'\x00' * len(decrypted_file)
del decrypted_file
    \end{minted}
    \caption{Uklanjanje sadržaja učitane datoteke i dekriptiranog DEK-a iz memorije.}
\end{listing}

\section{Revizijski trag}

Implementiran je nepromjenjiv revizijski trag s mogućnošću izvoza.
U aplikaciji se sve radnje korisnika, kao što su prijava, prijenos datoteke, promjena lozinke, izvoz audit loga ili drugo, bilježe.
To se obavlja u pozadini i ne smeta radu korisnika.
U bazi podataka zapisi revizijskog traga imaju sljedeći oblik:

\begin{figure}[H]
    \centering
    \includegraphics[width=1\textwidth]{slike/revizijski_trag_u_bazi.png}
    \caption{Revizijski trag u bazi podataka.}
\end{figure}

Na slici je vidljivo kako svaki zapis ima vlastiti sažetak (engl. \textit{hash}) i sažetak prethodnog zapisa, čime se ostvaruje \textbf{lanac sažetaka} (engl. \textit{hash chain}) i onemogućuje bilo kakva izmjena u prethodnim zapisima.

\subsection{Izvoz revizijskog traga}

Izvoz zapisa revizijskog traga moguće je obaviti na kartici "Izvoz audit logova" gdje je prije samog izvoza potrebno unijeti javni ključ osobe koja provjerava zapise:

\begin{figure}[H]
    \centering
    \includegraphics[width=1\textwidth]{slike/izvoz_audit_logova.png}
    \caption{Izvoz revizijskog traga.}
\end{figure}

Nakon izvoza stvara se datoteka s ekstenzijom \textit{.alogpkg} (\textit{audit log package}) koja sadrži enkriptirane zapise, enkriptiran simetrični ključ i digitalni potpis korisnika koji je izveo revizijski trag.
Također se nakon izvoza pojavljuje gumb "Kopiraj" koji kopira javni ključ osobe čiji se revizijski trag provjerava kako bi druga osoba mogla provjeriti digitalni potpis.

Aplikacija pri izvozu revizijskog traga obavlja sljedeće:
\begin{enumerate}
    \item Učitava zalijepljen javni ključ osobe koja će pregledavati revizijski trag.
    \item Generira novi nasumični simetrični ključ kojim će biti kriptirani zapisi revizijskog traga.
    \item Dohvaća sve zapise revizijskog traga iz baze podataka. \textbf{U slučaju da lanac sažetaka (engl. \textit{hash chain}) zapisa ne odgovara, u zapisima će biti vidljivo.}
    \item Kriptira zapise revizijskog traga koristeći simetrični ključ i AES algoritam.
    \item Kriptira novogenerirani simetrični ključ javnim ključem osobe koja će provjeravati zapise koristeći RSA algoritam.
    \item Digitalno potpisuje nekriptiran tekst zapisa revizijskog traga javnim ključem osobe čiji se zapisi izvoze.
    \item Upakiruje kriptirane zapise, kriptirani simetrični ključ i digitalni potpis u \textit{.alogpkg} datoteku.
\end{enumerate}

U slučaju da lanac sažetaka ne odgovara, to bi u zapisima izgledalo ovako:

\begin{listing}[H]
    \begin{minted}[frame=lines, linenos, breaklines]{text}
ID | Vrijeme | Poruka

ZAPISI SU MIJENJANI: Log zapisi nisu validni od zapisa s ID-om 5 - nije validan hash tog zapisa.

1 | 27.11.2025. 17:56:30 | Aplikacija je pokrenuta.
2 | 27.11.2025. 17:59:13 | Aplikacija je pokrenuta.
3 | 27.11.2025. 17:59:20 | Aplikacija je pokrenuta.
4 | 27.11.2025. 18:08:33 | Aplikacija je pokrenuta.
5 | 27.11.2025. 18:08:55 | Aplikacija NIJE pokrenuta.  <--- POTENCIJALNE IZMJENE OD OVOG ZAPISA
6 | 27.11.2025. 18:09:53 | Aplikacija je pokrenuta.
7 | 27.11.2025. 18:12:24 | Aplikacija je pokrenuta.
8 | 27.11.2025. 18:12:54 | Aplikacija je pokrenuta.
    \end{minted}
    \caption{Izmijenjeni zapisi revizijskog traga.}
\end{listing}

\subsection{Učitavanje revizijskog traga}

Učitavanje revizijskog traga moguće je obaviti na zaslonu "Pregled audit logova" na kojemu je prvo potrebno unijeti javni ključ osobe čiji se revizijski trag pregledava, zatim se klikom na gumb "Dalje" učitava datoteka revizijskog traga (\textit{.alogpkg}):

\begin{figure}[H]
    \centering
    \includegraphics[width=1\textwidth]{slike/ucitavanje_audit_logova.png}
    \caption{Učitavanje revizijskog traga.}
\end{figure}

Ako je učitana ispravna datoteka te su zapisi ispravni i digitalni potpis odgovarajući, moguće je vidjeti sve zapise revizijskog traga unutar same aplikacije:

\begin{figure}[H]
    \centering
    \includegraphics[width=1\textwidth]{slike/svi_zapisi_audit_logova.png}
    \caption{Svi zapisi revizijskog traga vidljivi unutar aplikacije.}
\end{figure}

Aplikacija pri učitavanju revizijskog traga obavlja sljedeće:
\begin{enumerate}
    \item Učitava zalijepljeni javni ključ osobe čiji se zapisi revizijskog traga provjeravaju.
    \item Učitava u memoriju sadržaj datoteke revizijskog traga (\textit{.alogpkg}).
    \item Dekriptira simetrični ključ za dekripciju zapisa revizijskog traga koristeći vlastiti privatni ključ.
    \item Dekriptira sadržaj revizijskog traga koristeći taj simetrični ključ.
    \item Provjerava digitalni potpis koristeći javni ključ osobe čiji se zapisi provjeravaju.
    \item Prikazuje sadržaj zapisa na zaslon.
\end{enumerate}

\chapter{Zaključak}

Digitalni sigurnosni trezori nisu uobičajeni sustavi za pohranu podataka. U njima datoteke nisu samo pohranjene, nego su i zaštićene mnoštvom složenih kriptografskih algoritama i sigurnosnim mehanizmima. Takvi sustavi su uglavnom implementirani pomoću Web tehnologija i nalaze se na internetu kako bi korisnicima omogućili kolaboraciju, dijelenje i pristup podacima s bilo koje lokacije. Mana takvih sustava je upravo to da su na mreži te su k puno više izloženi prijetnjama i napadima. Ovakvi trezori su već odavno u upotrebi u velikim organizacijama koje rukuju osjetljivim podacima kao što su financijski dokumenti, medicinski zapisi, pravni dokumenti itd. Dostupni su i komercijalno proizvodi za osobnu upotrebu iako su dosta skupi i ne baš toliko popularni kod običnih korisnika.

S druge strane, postoje klijentski sigurnosni trezori tj. aplikacije koje se mogu instalirati na računalo i zaštititi datoteke korisnika. Takvi sustavi sigurniji su od mrežnih napada i prijetnji jer nemaju taj vektor napada. Takvi sustavi su dosta rijetki i nisu toliko razvijeni jer nema velikog broja korisnika za njim. Često takvi alati i nisu javno dostupni ili su pak razvijeni za interne potrebe velikih organizacija.

U radu smo pokušali razviti jedan takav sustav koji bi zadovoljio potrebe malih korisnika. Najveća prednost ovog rada je taj što se datoteke u sustavu uništavaju i čiste tako da ako se i dođe do korisnikovog računala. datoteke neće biti dostupne. Poštovanjem standarda ISO 27001, pokušali smo dovesti sustav do visoke razine sigurnosti. Služio nam je kao vodič za implementaciju mehanizama unutar aplikacije. Bez njega, teško bi znali od kuda početi i što sve zapravo treba implementirati da bi sustav bio dovoljno siguran. Trenutna implementacija sadrži dosad najbolje kriprografske algoritme koji ne bi trebali biti probijeni neko duže vrijeme. Sigurnost sustava također ovisi i o korisniku koji ga koristi i o tome kako će on podesiti parametre svoje sigurnosne politike. Sustav nije savršen i uvjek postoji mogućnost poboljšanja, ali za sada je ovo odličan početak. 

\makebackmatter

\end{document}
