\documentclass[]{foi}
\usepackage{lipsum}
\usepackage[utf8]{inputenc}
\usepackage{float}

\vrstaRada{\seminar}
\predmet{Strate\v{s}ko upravljanje informacijskom sigurno\v{s}\'{c}u i privatno\v{s}\'{c}u}
\title{Klijentski kriptirani trezor datoteka s revizijskim tragom u Pythonu}

\author{Viktor Lovrić, Magdalena Markovinović, David Matijanić}
\spolStudenta{\musko}
\mentor{Izv. prof. dr. sc. Igor Tomičić}
\spolMentora{\musko}
\godina{2025}
\mjesec{Listopad}
\date{2025}
\status{redoviti}
\indeks{0016154953, 0016155896, 0016153844}
\smjer{Informacijsko i programsko inženjerstvo}
\titulaProfesora{}

\sazetak{Napisati (ono s maila translate)}

\kljucneRijeci{Napisati}

\begin{document}

\maketitle

\tableofcontents

\pagestyle{plain}

\chapter{Uvod}

Piše Megi

\section{ISO 27001}

Piše Megi

\chapter{Mehanizmi potrebni za implementaciju}

U ovom dijelu rada su opisani svi mehanizmi potrebni za implementaciju klijentskog kriptiranog trezora s revizijskim tragom koji je u skladu s ISO 27001 standardom.
U kasnijem dijelu rada su ovi mehanizmi pretvoreni u funkcionalne i nefunkcionalne zahtjeve te i implementirani.

\section{Slojevita enkripcija (engl. \textit{envelope encryption})}

Slojevita enkripcija je proces kriptiranja ključa drugim ključem\cite{google_envelope}.
Ključ za kriptiranje podataka (engl. \textit{data encryption key}), u nastavku teksta \textbf{DEK}, je simetrični ključ koji se koristi za kriptiranje i dekriptiranje podataka jedne datoteke jednog korisnika.
Za svaku datoteku se koristi zaseban DEK koji se nakon generiranja kriptira pri pohrani što zadovoljava zahtjev kriptiranja pri pohrani (engl. \textit{encryption at rest})\cite{google_envelope}.
\newline
Ključ za kriptiranje ključa (engl. \textit{key encryption key}), u nastavku teksta \textbf{KEK}, je asimetrični javni ključ koji koji se koristi za \textit{omotavanje} (engl. \textit{wrapping}) DEK-a, tj. kriptiranje DEK-a\cite{google_envelope}.
KEK se generira prilikom kreiranja korisničkog računa koristeći RSA (engl. \textit{Rivest-Shamir-Adleman}) ili ECC (engl. \textit{Elliptic Curve Cryptography}) algoritme i koristi se kako bi jedan KEK kriptirao više DEK-ova što eliminira potrebu za rotiranem ključeva za svaku datoteku, već je potrebno rotirati samo KEK.
Za generiranje DEK-a je preporućeno korištenje AES-256 (engl. \textit{Advanced Encryption Standard}) algoritma u GCM (engl. \textit{Galois Counter Mode}) načinu rada zbog njegove sigurnosti i brzine\cite{google_envelope}.
AES-256-GCM pruža i efikasnu provjeru integriteta podataka koristeći autentifikacijsku oznaku (engl. \textit{authentication tag}) kojom prilikom dekriptiranja provjerava integritet podataka.


\section{Kontrola pristupa i revizija}

Kontrola pristupa se sastoji od autorizacije i autentifikacije korisnika. Autentifikacija je proces u kojemu sustav provjerava identitet korisnika kako bi mu omogućio pristup\cite{ibm_access}, što se radi kroz jedno od 3 metode: nešto što korisnik zna (npr. lozinka), nešto što korisnik ima (npr. token) ili nešto što korisnik jest (npr. biometrijski podaci).
Autorizacija je proces određivanja razine pristupa resursima sustava na temelju korisničkog identiteta i njegovih dopuštenja\cite{ibm_access}. S obzirom da je naš sustav klijentski i zamišljen samo za jednog korisnika po uređaju, autorizacija se neće obavljati već će jedinstveni korisnik imati potpune ovlasti.
Odabrana metoda autentifikacije je lozinka iz koje se generira ključ koristeći neki od algoritama za derivaciju ključa poput PBKDF2 (engl. \textit{Password-Based Key Derivation Function 2}), scrypt ili Argon2 kako bi se taj ključ mogao pohraniti za autentifikaciju, a da pritom ostane zaštićen\cite{kdf}.
\newline
Revizija je proces pregleda i preispitivanja procesa i stanja kako bi se prosudila razina usklađenosti stanja s unaprijed definiranim kriterijima\cite{revizija}.
Kako bi se revizija mogla provesti, potrebno je voditi revizijski trag (engl. \textit{audit log}) svih važnih događaja i radnji u sustavu. Važna osobina revizijskog traga je njegova nepromjenjivost kako bi se spriječile zloupotrebe i manipulacije podacima\cite{audit}. Iz tog razloga se za validaciju ispravnosti revizijskog traga koriste mehanizmi poput digitalnog potpisa i lanca sažetaka (engl. \textit(hash chain)) gdje se svaki novi zapis povezuje sa sažetkom prethodnog zapisa, čime se osigurava da bilo kakva izmjena u prethodnim zapisima rezultira promjenom u sažetcima i time postaje vidljiva, što je ista tehnika koja se koristi u blockhain sustavima.
Još jedna važna karakteristika revizijskog traga je što on mora biti izvoziv (engl. \texit(exportable)) kako bi se mogao analizirati izvan samog sustava, npr. u slučaju inspekcije od strane treće strane.
U tom je slučaju potrebno revizijski trag kriptirati javnim ključem osobe koja će vršiti inspekciju kako bi se osigurala povjerljivost podataka.

\section{Životni ciklus ključa}

\section{Sigurnosno dijeljenje}

\section{Sigurnosno brisanje}

\section{Mehanizam zaključavanja ili odgode}

\chapter{Tehnologije za implementaciju}

Piše David

\chapter{Funkcionalnosti}

Piše David

\makebackmatter

\end{document}
